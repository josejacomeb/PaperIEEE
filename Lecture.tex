\documentclass[10pt,a4paper]{proc}
\usepackage[utf8]{inputenc}
\usepackage[english]{babel}
\usepackage{amsmath}
\usepackage{amsfonts}
\usepackage{amssymb}
\author{José Jácome}
\title{Lecture for Conference}
\begin{document}
\section{Presentation}
Ladies and Gentlemen, my name is Jos\'e J\'acome, I'm a Mechatronics Engineering student and I'm representing the Universidad de las Fuerzas Armadas ESPE as well as the co-authors of this paper, Andrea C\'ordova and Mauricio Barreno. I will present to you our paper titled "Precise Weed and Maize Classification through Convolutional Neuronal Networks", so let's start...
\section{Introduction}
So first, lets begin talking about Maize, it's one of the most important crops in the world. This Crop can be significantly affected by weeds, causing a yield reduction of up to 5000 Kg/Ha, so it is necessary to develop solutions for this particular problem and Precision Agriculture with Robotics have a great potential for solving this kind of issues. It's application is possible due to the fact that AI has reached near-to-human precision
\section{Purpose of the present study}
The purpose of this study was to develop a Maize and Weed "Classifier" of Maize and Weed, so firstly we:
Obtained Sample images to conform a maize dataset and Make an Algorith of Image Processing to Segment Samples. Afterwards we tested the accuracy in different network architectures of Convolutional Neural Networks for Maize and Weed Classification. Later it was necessary to Benchmark the appropriate network architecture  and finally optimize the network to run in a specific hardware. 
\section{Used Hardware and Software}
To obtain the desired result, we have chosen the following hardware: A popular embedded system: Raspberry Pi 3 with its camera to get the samples, a CPU Core i7 6Th Generation and a Nvidia Graphics Card the GTX950M. \\
And also we used free software to develop the elements of the net, first an image proccessing Library OpenCV, Caffe Framework to develop general-purpose convolutional (Fully Compatible with Python and C++) and Linux OS such as Ubuntu for the PC and Pixel for the Raspberry Pi.

\section{Image Processing}
With the help of OpenCV, we developed an algorithm. A firs step was to Acquire a RGB image from the Pi Camera, centered and in a vertical position on reference to the plant, next to we normalized the Green Channel to avoid lights and shadows of the Image, afterwards we subtracted the Green Channel from the Blue and Red Channel with the next equation described by Wang. After this, the image needed to be thresholded through the OTSU Method. Following this, it was necessary to Detect the contours and crop the image. furthermore, to get a maximum detail we masked the image. In the end we show some examples of the image processing process. There are common plants in the Ecuadorin Highlands such as Ortiga, Allpaquinua and Malva that fall under weed classification.
\section{Dataset}
The dataset was conformed by samples took in Pillaro-Tungurahua-Ecuador, a great producer of many crops such as maize, the samples were obtained in crops at stage V3-V7(3-7 leaves). Following the procedure established by Sladojevic, we rotated images every 30º to improve the detection of plants in any position. Following this, the fifth part of the total images were chosen randomly to validate the neural network training and we obtained following chart. 
\section{Convolutional Neural Networks}
Let's talk about CNN, It's a highly accurate method for image classification, aclas of deep, feed-forward artificial NN. it has been widely tested in classification of plants and has multiple architectures and applications. We  can see a cNET architecture here, with the main layers such as Convolution, Pooling and Fully Connected.
\section{Tested Architectures}
Four architectures were chosen, the first two were chosen from the Caffe Zoo Model, a open resource of CNN Architectures, and the last two were chosen from the paper by Ciro Potena and others, sNET and cNET(beetroot). So, the following results were obtained, the net with the best performance was cNET with 96.4\% of accuracy and 13.72\$ of Loss. 
\section{cNET Performance}
cNET has 6 million of filters, so cNET can be improved by decresing the number of filtres. Also Images can be batched and also Caffe ca be Multithreaded, and both nets were trained with 9000 iterations, so the final result was better because the dataset images don't have too many features of shape and color to process. 
\section{Estimated performance of cNET 16 filters}
well, like I mentioned before, the nets in Caffe can be tuned, so we have performance results with these specific parameters. It was considereded that 18 plants can be found in a single image of a row of crop.
\section{Conclusion}
\begin{itemize}
\item cNET had the best performance in classification of maize and weed
\item The reduction of the number of filters decreased the processing time and increased the network accuracy 
\item GPU showed the best results, but with Multithreading and Batching CPU and Raspberry Pi its processing time can be improved 
\item Due to the limitations of the Raspberry Pi, it can't be used to classify in real time, but a Neural Module(such as Intel Movidius) can improve its performance \end{itemize}
Thanks!
\end{document}