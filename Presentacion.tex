\documentclass[10pt,a4paper]{beamer}
\usepackage[utf8]{inputenc}
\usepackage[english]{babel}
\usepackage{amsmath}
\usepackage{amsfonts}
\usepackage{amssymb}
\usepackage{makeidx}
\usepackage[style=verbose]{biblatex}
\usepackage{graphicx}
\usepackage{makecell}
\usepackage{verbatim}
\usefonttheme{professionalfonts} % fuentes de LaTeX
\usetheme{Szeged}
\usecolortheme{beaver}
\setbeamercovered{transparent}
\addbibresource{bibliografiapaper.bib}
\begin{document}
%\font\myfont=cmr12 at 20pt
%\title{{\myfont Precise Weed and Maize Classification through Convolutional Neuronal Networks}}
%\author{{\bf C\'ordova Andrea\inst{1}, Barreno Mauricio, J\'acome Jos\'e }\\
%\institute{
%\inst{1}%
 % Universidad de las Fuerzas Armadas ESPE\\
 % Latacunga - Ecuador}
%{ 2nd IEEE Ecuador Technical Chapters Meeting}
%\vspace*{1cm}}
%\date{October 2017}
\title[Precise Weed and Maize Classification through Convolutional Neuronal Networks] %optional
{Precise Weed and Maize Classification through
Convolutional Neuronal Networks}
 
 
\author[C\'ordova Andrea, Barreno Mauricio and J\'acome Jos\'e] % (optional, for multiple authors)
{C\'ordova Andrea, Barreno Mauricio and J\'acome Jos\'e}
 
\institute[Universidad de las Fuerzas Armadas ESPE] % (optional)
{
  \inst{1}%
  Departamento de Energ\'ia y Mec\'anica\\
  Universidad de las Fuerzas Armadas ESPE
}
 
\date[Ecuador] % (optional)
{2nd IEEE Ecuador Technical Chapters Meeting, October 2017}
\logo{\includegraphics[height=1.5cm]{espe.png}}
\frame{\titlepage}
%-----------------------------
\begin{frame}
\frametitle{Presentation Outline}
\tableofcontents
%\newpage
\end{frame}
%-----------------------------
\AtBeginSection[]
{
  \begin{frame}
    \frametitle{Table of Contents}
    \tableofcontents[currentsection]
  \end{frame}
}
\section{Introduction}
\begin{frame}
\frametitle{Introduction}
\textbf{Introduction}
\begin{itemize}
\item Maize is one of the most important \textbf{food or crop?} of the world.
\item Weed can affect the maize crop up to 5000 Kg/Ha.\footnote{suarez2005distintos} %Las malas hierbas en el maiz pueden afectan hasta 5000 Kg/Hectaria de produccion.
\item Robotics has presented a huge advance in Precision Agriculture.%La robotica esta presentando grandes avances en la agricultura de precision.
\item Artificial Intelligence reached near-to-human precision.%Inteligencia artificial cada vez se acerca mas a la inteligencia humana.
\end{itemize}
\textbf{The Propose of the present study}
\begin{itemize}
\item Obtain samples to conform a dataset%Obtener muestras(imagenes) para conformar un dataset.
\item Segment samples%Segmentar las muestras.
\item Test diffent network architectures of Convolutional Neural Networks for classify Maize and Weed%Probar diferentes arquitecturas de Redes Neuronales Convolucionales para entrenar la red.
\item Benchmark the best architure to analyse time of processing %Hacer Benchmark en diferentes hardwares para analizar tiempos de procesamiento.
\item Optimize the time of processing of the net%Optimizar la red.
\end{itemize}
\end{frame}
%-----------------------------
\AtBeginSection[]
{
  \begin{frame}
    \frametitle{Table of Contents}
    \tableofcontents[currentsection]
  \end{frame}
}
\section{Hardware and Software Used}
\begin{frame}
\frametitle{Hardware and Software Used}
\textbf{Hardware}
\begin{enumerate}
\item Raspberry Pi 3.
\item Pi camera V2.1.
\item Nvidia graphic Card GTX950M.
\end{enumerate}
\textbf{Software}
\begin{enumerate}
\item OpenCV Library
\item Caffe framework
\item Ubuntu 16.04
\item PIXEL Distribution derived from Debian.
\end{enumerate}
\end{frame}
%-----------------------------
\AtBeginSection[]
{
  \begin{frame}
    \frametitle{Table of Contents}
    \tableofcontents[currentsection]
  \end{frame}
}
\section{Image Processing}
\begin{frame}
\begin{itemize}
	\item Acquire  an RGB image through RPi Camera v2.1
	\item Detect contours and crop image to the contour
	\item Normalize Green Channel and then $S = 2*G - R - B$ \footcite{wang2013path}
	\item OTSU Thresholding
	\item Mask image
\end{itemize}
\frametitle{Image Processing}
	\begin{figure}[h]
	\centering
	\includegraphics[width=3.5 in]{procesamiento}
	\caption{The process of image processing,\textbf{ por lo pronto}}
	\label{figure4}
	\end{figure}
\end{frame}
%-----------------------------
\AtBeginSection[]
{
  \begin{frame}
    \frametitle{Table of Contents}
    \tableofcontents[currentsection]
  \end{frame}
}
\section{Dataset}
\begin{frame}
\frametitle{Dataset}
%{Dataset Description}
\begin{itemize}
\item Samples obtained in Pillaro-Tungurahua-Ecuador
\item Images obtained in its initial stage(3-7 leaves) . %La imagenes fueron obtenidas encampos de maiz en su estapa inicial,( plantas con 3 a 7 hojas) 
\item Rotated images every 30º to improve detection of plants\footcite{sladojevic2016deep}%Imagenes rotadas cada 30 grados para mejorar la deteccion de las planta.
\item 1/5 of the total images chosed randomly to validate training%1/5 del total de las imagenes al azar fueron usadas para la etapa de validacion
\end{itemize}
\begin{table}[h!]
\renewcommand{\arraystretch}{1.3}
\caption{Dataset distribution of each class}
\label{table:1}
\centering
\begin{tabular}{| c c c |} 
 \hline
 \textbf{Images} & \textbf{Maize} & \textbf{Weed}  \\ [1ex] 
 \hline
 Original  & 2835 & 880 \\ 
 Rotated & 34222 & 10762 \\ 
 Training & 25695 & 8560 \\
 Validation & 8325 & 2000 \\
 \hline
\end{tabular}
\end{table}
\end{frame}

%-----------------------------
\AtBeginSection[]
{
  \begin{frame}
    \frametitle{Table of Contents}
    \tableofcontents[currentsection]
  \end{frame}
}
\section{Convolutional Neural Networks}
\begin{frame}
\frametitle{Convolutional Neural Networks(CNN)}
\begin{itemize}
	\item Highly accurate method for image classification
	\item A class of deep, feed-forward artificial neural networks
	\item Tested on classification of plants, \footcite{cheng2015feature} \footcite{potena2016fast} \footcite{di2016automatic}
	\item Multiple architectures and applications
\end{itemize}
	\begin{figure}[h]
	\centering
	\includegraphics[width=3.5 in]{arquitectura}
	\caption{Normal architecture in a Convolutional Neural Network}
	\label{figure4}
	\end{figure}
\end{frame}
\AtBeginSection[]
{
  \begin{frame}
    \frametitle{Table of Contents}
    \tableofcontents[currentsection]
  \end{frame}
}
\section{Architectures Tested}
\begin{frame}
\frametitle{Architectures tested}
\begin{itemize}
	\item LeNET and AlexNet(Caffe Zoo Model) \
	\item cNET and sNET \footnote{potena2016fast}
	\item 3000 iterations in each training
\end{itemize}
\begin{table}[h!]
\centering
\renewcommand{\arraystretch}{1.2}
\caption{Comparison of the 4 types of CNN in training the dataset}
\label{table:2}
\begin{tabular}{|l c c c c|} 
 \hline
 \textbf{Parameters }& \textbf{LeNet} & \textbf{AlexNet} & \textbf{cNET} & \textbf{sNET} \\ [0.75ex] 
 \hline
 Input size of images & 32x32 & 64x64 & 64x64 & 64x64 \\ 
 Layers numbers & 9 & 11 & 8 & 4\\
 Number of parameters & 652500 & 20166688 & 6421568 & 135872 \\ 
 Accuracy(\%) & 86.48 & 93.86 & 96.4 & 80.4 \\
 Loss(\%) & 32.80 & 15.32 & 13.72 & 15.32 \\ [1ex] 
 \hline 
\end{tabular}
\end{table}
\end{frame}
%-----------------------------
\AtBeginSection[]
{
  \begin{frame}
    \frametitle{Table of Contents}
    \tableofcontents[currentsection]
  \end{frame}
}
\section{Tuning cNET}
\begin{frame}
\frametitle{cNET Performance}
\begin{enumerate}
\item cNET can be improved by decrease the number of filters
\item Images can be batched and also Caffe can be multithreaded
\item Both nets were trained with 9000 iterations
\end{enumerate}
\begin{table}[h]
\centering
\renewcommand{\arraystretch}{1.2}
\caption{Comparison between cNET of 16 and 64 filters}
\label{table:3}
\begin{tabular}{|l c c |} 
 \hline
 \textbf{Parameters} & \textbf{cNET 16 filters} & \textbf{cNET 64 filters} \\ [0.75ex] 
 \hline
 Number of parameters & 1651376 & 6421568 \\ 
 Accuracy(\%) & 97.26 & 96.40 \\
 Loss(\%) & 8.39 & 13.72 \\ [1ex] 
 \hline 
\end{tabular}
\end{table}
\end{frame}
%-----------------------------
\AtBeginSection[]
{
  \begin{frame}
    \frametitle{Table of Contents}
    \tableofcontents[currentsection]
  \end{frame}
}
\section{Presuming performance of cNET 16 filters}
\begin{frame}
\frametitle{Presuming performance of cNET 16 filters}
\begin{itemize}
	\item A dataset of test with 202 images of each class was used
	\item In a single image it can be found 18 plants to classify
\end{itemize}
\begin{table}[h!]
\centering
\renewcommand{\arraystretch}{1.2}
\caption{Test of complete image classification in FPS}
\label{table:6}
\begin{tabular}[c c c c]{|p{1.8 cm} p{1.5cm} p{2.1cm} p{2.3cm}|} 
 \hline
 \textbf{Parameter} &\textbf{GPU } & \textbf{CPU} & \textbf{Raspberry Pi} \\ 
 \hline
  \textbf{Method} &\textbf{One Core} & \textbf{Multithreading} & \textbf{Multithreading} \\ 
 \hline
 Time(s) & 0.0171 & 0.196 & 2.714 \\ [0.95ex]
 FPS & 58.47 & 5.08 & 0.36 \\ [0.95ex]
 \hline
\end{tabular}
\end{table}
\end{frame}
%--------------
\AtBeginSection[]
{
  \begin{frame}
    \frametitle{Table of Contents}
    \tableofcontents[currentsection]
  \end{frame}
}
\section{Conclusion}
\begin{frame}
\frametitle{Conclusion}
\begin{itemize}
\item cNET showed the better results in classification of maize and weed
\item The reduce of the number of filters allow to decrease the processing time and increase the accuracy of the net
\item GPU showed the best results but with Multithreading and Batching CPU and Raspberry Pi can improve its time of processing 
\item Due to the limitations of the Raspberry Pi, it can't be used to classify in real time, but a Neural Module(such as Intel Movidius) can improve that result
\end{itemize}
\end{frame}
%--------------
\large
\begin{frame}
%\frametitle{The Propose of the present study}
\begin{center}
Thanks!
\end{center}
\end{frame}

%-----------------------------
%\bibliographystyle{plain}
%\bibliography{bibliografiapaper}
\end{document}
